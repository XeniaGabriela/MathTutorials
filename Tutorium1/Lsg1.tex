\documentclass[12pt]{article}

\usepackage{amsmath,amsthm,amsxtra,amssymb}
\usepackage{multicol}
\usepackage{rotating}
\usepackage{moreverb}
\usepackage{epsfig}

\newtheorem{satz}{Satz}
\newtheorem{thm}[satz]{Theorem}
\newtheorem{exercise}[satz]{Aufgabe}

\begin{document}

   \pagestyle{empty}
   \parindent 0cm
   \begin{minipage}{14cm}
   \footnotesize{\textbf{HOCHSCHULE} \hfill Dipl.-Math. Xenia Bogomolec\\
  \textbf{HANNOVER}\\
   UNIVERSITY OF\\
   APPLIED SCIENCES\\
   AND ARTS
   }
   \end{minipage}
   \vspace{1.0cm}

   \begin{center}
   {\Large \bf Br\"uckenkurs Mathematik} \\
     \vspace{0.5cm}
     {\large L\"osungen zum \"Ubungsblatt 1}  \\

  \end{center}
   \vspace{0.5cm}
   \normalsize
   \parindent0cm
   
  \begin{exercise}
  Multiplizieren Sie folgende Ausdr\"ucke aus und fassen Sie sie dann zusammen:
  \begin{enumerate}
  \item[(a)] $ a(b+c(a-b)) - b(a+c(1+a)) - c(b-a(c-b)) = ac^2+ca^2-3abc-2bc$
  \item[(b)] $ (a+b+c)(a-b-c)+(a+b-c)(a-b+c) = 2a^2-2b^2-2c^2$
  \end{enumerate}
   \end{exercise}

    \vspace{0.1cm}

  \begin{exercise}
  Fassen Sie folgende Ausdr\"ucke geschickt zusammen:
  \begin{enumerate}
  \item[(a)] $(4x-2y)^2+16y(x-5y) = 4(4x^2-19y^2)$
  \item[(b)] $3(\ln{x}+x^2)+x(\ln{x}+9) = (\ln{x} + 3x)(3+x)$
  \item[(c)] $(3\cos{x}+y)(3\cos{x}-y)+(y+z)(y-z) = 9\cos{x}^2-z^2$ 
  \end{enumerate}
   \end{exercise}

    \vspace{0.2cm}
   
   \begin{exercise}
  Faktorisieren Sie mit Hilfe der binomischen Formeln oder des Satzes von Vieta:
  \begin{enumerate}
  \item[(a)] $x^2-169 = (x+13)(x-13)$ 
  \item[(b)] $x^2+9x-22 = (x+11)(x-2)$ 
  \item[(c)] $16x^2y^2+24xy^2z+9y^2z^2 = y^2(4x+3z)^2$
  \item[(d)] $5y\sin^2{x}+20y\sin{x}+20y = 5y(\sin{x}+2)^2$
  \end{enumerate}
   \end{exercise} 

   \vspace{0.2cm}

   \begin{exercise}
  K\"urzen Sie die Br\"uche:
  \begin{multicols}{2}
  \begin{enumerate}
  \item[(a)] $\frac{4(x^2-16)(x-5)}{12(x-4)(x^2-10x+25)} = \frac{(x+4)}{3(x-5)}$
  \item[(b)] $\frac{3a^2b^2}{9b^2} = \frac{a^2}{3}$
  \end{enumerate}
  \end{multicols}

  \begin{multicols}{2}
  \begin{enumerate}
  \item[(c)] $\frac{(\sin{x})^2 + 17 \sin{x}}{(\sin{x})^2} = 1 + \frac{17}{\sin{x}}$
  \item[(d)] $\frac{4(\ln{x} + 3)-(2-\ln{x})}{4 + (\ln{x})^2 + 4\ln{x}} = \frac{5}{\ln{x} + 2}$
  \end{enumerate}
  \end{multicols}
   \end{exercise}

    \vspace{0.2cm}

   \begin{exercise}
  Addieren oder subtrahieren Sie folgende Br\"uche und k\"urzen Sie das Ergebnis:
  \begin{multicols}{2}
  \begin{enumerate}
  \item[(a)] $\frac{3}{5} + \frac{2}{9} = \frac{37}{45}$
  \item[(b)] $\frac{b}{a} + \frac{a}{b} = \frac{a^2 + b^2}{ab}$
  \item[(c)] $\frac{x+1}{x-2} - \frac{3}{x+2} = \frac{x^2+8}{x^2-4}$
  \item[(d)] $\frac{3\cos{x} + (\sin{x})^2}{\sin{x}\cos{x}} - \frac{\sin{x}}{\cos{x}} = \frac{3}{\sin{x}}$
  \end{enumerate}
  \end{multicols}
   \end{exercise}

    \vspace{0.2cm}

   \begin{exercise}
  Multiplizieren oder dividieren Sie folgende Quotienten und k\"urzen Sie das Ergebnis:
  \begin{multicols}{2}
  \begin{enumerate}
  \item[(a)] $\frac{(a+b) \cdot a}{b} \cdot \frac{(a+b)}{a^2+ab} = \frac{a+b}{b}$
  \item[(b)] $\frac{64}{15} \cdot \frac{105}{96} = 4\frac{2}{3}$
  \item[(c)] $\frac{2x+8}{9} \div \frac{9}{x-4} = \frac{2x^2-32}{81}$
  \item[(d)] $\frac{x(x-2)}{6(x+2)} \div \frac{(x^2-4)}{4} = 
  \frac{2x}{3(x+2)^2}$
  \end{enumerate}
  \end{multicols}
   \end{exercise}

    \vspace{0.2cm}

   \begin{exercise}
  Vereinfachen Sie folgende Quotienten:
  \begin{multicols}{2}
  \begin{enumerate}
  \item[(a)] $\frac{a}{\frac{a}{b}} = b$
  \item[(b)] $\frac{\frac{4 \ln{x}}{a}}{\frac{2a}{(\ln{x})^2}} = \frac{2(\ln{x})^3}{a^2}$
  \item[(c)] $\frac{\frac{x+1}{x+3}}{\frac{x+1}{x-3}-\frac{x+1}{x+3}} = \frac{x-3}{6}$
  \end{enumerate}
  \end{multicols}
   \end{exercise}

    \vspace{0.2cm}

    \begin{exercise}
  Vereinfachen und k\"urzen Sie die Br\"uche:
  \begin{enumerate}
  \item[(a)] $\frac{5(x+6)^8 (x-7)^{11}}{5(x+6)^3 (x-7)^{13}} = \frac{(x+6)^5}{(x-7)^2}$
  \item[(b)] $(\frac{a^2b^3c^4}{3(bc)^3})^2 = \frac{a^4c^2}{9}$
  \end{enumerate}
   \end{exercise}
   
 
\end{document}
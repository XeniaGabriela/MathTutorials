\documentclass[12pt]{article}

\usepackage{amsmath,amsthm,amsxtra,amssymb}
\usepackage{multicol}
\usepackage{rotating}
\usepackage{moreverb}
\usepackage{epsfig}

\newtheorem{satz}{Satz}
\newtheorem{thm}[satz]{Theorem}
\newtheorem{exercise}[satz]{Aufgabe}

\begin{document}

   \pagestyle{empty}
   \parindent 0cm
   \begin{minipage}{14cm}
   \footnotesize{\textbf{HOCHSCHULE} \hfill Dipl.-Math. Xenia Bogomolec\\
  \textbf{HANNOVER}\\
   UNIVERSITY OF\\
   APPLIED SCIENCES\\
   AND ARTS
   }
   \end{minipage}
   \vspace{1.0cm}

   \begin{center}
     {\Large \bf Br\"uckenkurs Mathematik zum Wintersemester 2015/2016} \\
     \vspace{0.5cm}
     {\large L\"osungen zum \"Ubungsblatt 5}  \\

  \end{center}
   \vspace{0.5cm}
   \normalsize
   \parindent0cm
   
  \begin{exercise}
  Bestimmen Sie die L\"osungsmengen folgender Gleichungen:
  \begin{enumerate}
  \item[(a)] $x^2+5x-14=(x-2)(x+7)=0 \,\Rightarrow \mathcal{L} = \{2,-7\}$
  \item[(b)] $\sqrt{81x^2+36x+4}=\sqrt{(9x+2)^2}=11 $\\
             $\Leftrightarrow 9x+2=11 \,\, oder \,-9x-2=11$
             $\,\Rightarrow \mathcal{L} = \{1,-\frac{13}{9}\}$
  \item[(c)] $3^x = 243 \,\Rightarrow x=5$
  \item[(d)] $a^{x+15}=a^8$, \,Exponentenvergleich liefert\, $x+15=8\Rightarrow x=-7$
  \item[(e)] $v(v^{x-3})^{x+2} = v^4(v^{3x+1})^{x-3}\Leftrightarrow
             v(v^{(x-3)(x+2)}) = v^4(v^{(3x+1)(x-3)})\Leftrightarrow
             v^{x^2-x-6}=v^{3x^2-8x+1}$,
            \, Exponentenvergleich liefert\, $x^2-x-6=3x^2-8x+1$ \\
             $\Rightarrow 2x^2-7x+5=0 \Leftrightarrow x^2-\frac{7}{2}x+\frac{5}{2}=0$\\
             $\Rightarrow x_{1,2}=\frac{7}{4}\pm\sqrt{\frac{49}{16}-\frac{5}{2}}=\frac{7}{4}\pm\sqrt{\frac{49-40}{16}}=\frac{7}{4}\pm\sqrt{\frac{9}{16}}=\frac{7}{4}\pm\frac{3}{4}$
             $\,\Rightarrow \mathcal{L} = \{1,\frac{5}{2}\}$
  \item[(f)] $6 + 2\lg{x} = 2 \Leftrightarrow 2\lg{x} = -4 \Leftrightarrow 
            \lg{x}=\log_{10}x = -2 \,\Rightarrow x=10^{-2}=\frac{1}{100}$\\
  \item[(g)] $\lg{\sqrt[3]{4x}} = \frac{1}{2}\Leftrightarrow \frac{1}{3}(\lg{4}+\lg{x})=\frac{1}{2}\Leftrightarrow \lg{x}=\log_{10}x=\frac{3}{2}-\lg{4}\,\Rightarrow x=10^{(\frac{3}{2}-\lg{4})}$
  \end{enumerate}
   \end{exercise}

   \vspace{2cm}

  \begin{exercise}
  Suchen Sie nach Nullstellen folgender Polynomfunktionen, indem Sie sie (wenn m\"oglich) in Linearfaktoren zerlegen:\\
  Frage: Ist das f\"ur jedes Polynom m\"oglich?\\
  Hinweis: Nullstellen m\"ussen Faktoren des konstanten Gliedes sein.
  \begin{enumerate}
  \item[(a)] $f(x) = x^3-2x^2-5x+6:$ \\
             M\"ogliche Nullstellen sind $\pm1,\pm2,\pm3,\pm6$.\\
             Teste Polynomdivision f\"ur $1$ als Nullstelle: \\
             $(x^3-2x^2-5x+6)\div(x-1)=x^2-x-6$\\
             Geht ohne Rest auf, also ist $1$ eine Nullstelle von $f(x)$!\\
             $x^2-x-6=(x+2)(x-3)$ laut Vieta\\
             $\Rightarrow 1,-2$ und $3$ sind Nullstellen von $f(x)$ und $x^2-2x^2-5x+6$ zerf\"allt in Linearfaktoren.
  \item[(b)] $f(x) = 2x^4+2x^3+2x^2+2x = 2x(x^3+x^2+x+1):$ \\
             Wegen dem Faktor $2x$ haben wir die Nullstelle $x=0$,
             andere m\"ogliche Nullstellen sind $\pm1$.\\
             Teste Polynomdivision f\"ur $1$ als Nullstelle: \\
             $(x^3+x^2+x+1)\div(x-1)=x^2+2x+3+\ldots$\\
             Geht nicht ohne Rest auf, also ist $1$ keine Nullstelle von $f(x)$!\\
             Teste Polynomdivision f\"ur $-1$ als Nullstelle: \\
             $(x^3+x^2+x+1)\div(x+1)=x^2+1$\\
             Geht ohne Rest auf, also ist $-1$ eine Nullstelle von $f(x)$!\\
             Aber $x^2+1=0$ hat keine L\"osung in $\mathbb{R}$.\\
             $\Rightarrow -1$ und $0$ sind Nullstellen von $f(x)$ und $2x^4+2x^3+2x^2+2x$ zerf\"allt nicht in Linearfaktoren. 
  \end{enumerate}
   \end{exercise}
   
   \begin{exercise}
  L\"osen Sie die folgenden Gleichungen und pr\"ufen Sie, f\"ur welche $x$ sie definiert sind:
  \begin{enumerate}
  \item[(a)] $\frac{4}{x^2-4}+\frac{1}{x-2} = \frac{2x-4}{2(x+2)}:$ \\         
             $\frac{4+(x+2)}{x^2-4}=\frac{(x-2)^2}{x^2-4}\Leftrightarrow x+6=x^2-4x+4
             \Leftrightarrow x^2-5x-2 = 0$\\
             p-q-Formel ergibt $x_{1,2} = \frac{5}{2}\pm\sqrt{\frac{25}{4}+2}=\frac{5\pm\sqrt{33}}{2}$

  \item[(b)] $\sqrt{\frac{4}{9}x^2+\frac{4}{3}x+1}+2x=7:$ \\ 
             $\sqrt{(\frac{2}{3}x+1)^2}+2x=7 \Leftrightarrow (\frac{2}{3}x+1)+2x=7\Leftrightarrow \frac{2}{3}x+2x=6 \Leftrightarrow \frac{8}{3}x = 6 \Leftrightarrow x=\frac{18}{8}=\frac{9}{4}$
             (Wurzeln sind positiv definiert)
  \end{enumerate}
   \end{exercise} 

   \begin{exercise}
  L\"osen Sie die Gleichungen mit dem Substutionsverfahren und pr\"ufen Sie die L\"osungen:
  \begin{enumerate}
  \item[(a)] $x^6-16x^3=-64:\,\, y:=x^3$, dann ist $y = 8$ und $x=\sqrt[3]{8}=2$
  \item[(b)] $e^x+e^{-x}=2:\,\, y:=e^x$, dann ist $y = 1$ und $x=0$
  \item[(c)] $(\ln{x})^2-9\ln{x}=-20:\,\, y:=\ln{x}$, dann ist $y\in\{4,5\}$ und $x\in\{e^4,e^5\}$
  \end{enumerate}
   \end{exercise}

   \begin{exercise}
  Bestimmen Sie jeweils den Definitionsbereich von $f(x)$, die Umkehrfunktion $f^{-1}(x)$ und den Definitionsbereich von $f^{-1}(x)$:
  \begin{enumerate}
  \item[(a)] $f(x)=6x+3:$\\ 
             $y=6x+3 \Leftrightarrow x = \frac{y-3}{6} \Rightarrow f^{-1}(x) = \frac{x-3}{6}$\\
             $D_f = \mathbb{R}, D_{f^{-1}} = \mathbb{R}$
  \item[(b)] $f(x)=\frac{1}{25}x^2:$\\ 
             $y=\frac{1}{25}x^2 \Leftrightarrow 25y=x^2 \Leftrightarrow  x = \pm 5\sqrt{y} \Rightarrow f^{-1}(x) = \pm 5\sqrt{x}$\\
             $D_f = \mathbb{R}$\\
             \\
             Getrennte Betrachtung f\"ur die Umkehrfunktionen:\\
             $f^{-1}(x) = 5\sqrt{x}$ f\"ur $x\in {\mathbb{R}_0}^+, D_{f^{-1}} = {\mathbb{R}_0}^+$\\
             $f^{-1}(x) = -5\sqrt{x}$ f\"ur $x\in \mathbb{R}^-, D_{f^{-1}} = \mathbb{R}^+$\\
             \\
             Laut Definition einer Funktion darf man jedem $x$ nur ein $y$ zuordnen. Eine um $90^\circ$ im Uhrzeigersinn gedrehte Parabel erf\"ullt dieses Kriterium nicht, eine halbe Parabel jedoch schon.
  \item[(c)] $f(x)=\frac{x-1}{x+4}:$\\ 
             $y(x+4)=x-1 \Leftrightarrow xy-x=-1-4y \Leftrightarrow  x(y-1) = -1-4y \Leftrightarrow x=\frac{4y+1}{1-y}\Rightarrow f^{-1}(x) = \frac{4x+1}{1-x}$\\
             $D_f = \mathbb{R} \setminus \{-4\}, D_{f^{-1}} = \mathbb{R} \setminus \{1\}$
  \end{enumerate}
   \end{exercise}

   \begin{exercise}
  Bestimmen Sie die L\"osungsmengen der Ungleichungen und der Gleichung:
  \begin{enumerate}
  \item[(a)] $x+2<5x-8$:\\ 
             $4x>10 \Leftrightarrow x>\frac{5}{2} \Rightarrow L = (\frac{5}{2},\infty]$
  \item[(b)] $(x-3)^2<4$:\\ 
             $ -2<x-3<2 \Leftrightarrow 1<x<5 \Rightarrow L = (1,5)$
  \item[(c)] $x^2+3\leq 2$:\\ 
             $ x^2 \leq -1 \Rightarrow L = \emptyset$
  \item[(d)] $\vert \frac{x}{5}-2 \vert \leq 3$:\\ 
             $-3 \leq \frac{x}{5}-2 \leq 3 \Leftrightarrow -1 \leq \frac{x}{5} \leq 5\Leftrightarrow -5 \leq x \leq 25$\\
             $\Rightarrow L =[-5,25]$
  \item[(e)] $\vert x^2-2x \vert = 1$:\\ 
            $ x^2-2x = 1 \Leftrightarrow x^2-2x+1 = 2 \Leftrightarrow (x-1)^2 = 2$, also $x = 1+\sqrt{2}$\\
            $ x^2-2x = -1 \Leftrightarrow x^2-2x+1 = 0 $, also $x = 1$\\
            $\Rightarrow L =\{1,1+\sqrt{2}\}$
  \end{enumerate}
   \end{exercise}


\end{document}
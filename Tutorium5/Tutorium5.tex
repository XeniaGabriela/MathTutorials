\documentclass[12pt]{article}

\usepackage{amsmath,amsthm,amsxtra,amssymb}
\usepackage{multicol}
\usepackage{rotating}
\usepackage{moreverb}
\usepackage{epsfig}

\newtheorem{satz}{Satz}
\newtheorem{thm}[satz]{Theorem}
\newtheorem{exercise}[satz]{Aufgabe}

\begin{document}

\pagestyle{empty}
\parindent 0cm
\begin{minipage}{14cm}
  \footnotesize{\textbf{HOCHSCHULE} \hfill Dipl.-Math. Xenia Bogomolec\\
  \textbf{HANNOVER}\\
  UNIVERSITY OF\\
  APPLIED SCIENCES\\
  AND ARTS
  }
\end{minipage}
\vspace{1.0cm}

\begin{center}
  {\Large \bf Br\"uckenkurs Mathematik} \\
  \vspace{0.5cm}
  {\large \"Ubungsblatt 5 (Gleichungen und Ungleichungen)}  \\
\end{center}
\vspace{0.5cm}
\normalsize
\parindent0cm

\begin{exercise}
  Bestimmen Sie die L\"osungsmengen folgender Gleichungen:
  \begin{multicols}{2}
    \begin{enumerate}
      \item[(a)] $x^2+5x-14=0$
      \item[(b)] $\sqrt{81x^2+36x+4}=11$
      \item[(c)] $3^x = 243$
      \item[(d)] $a^{x+15}=a^8$
      \item[(e)] $v(v^{x-3})^{x+2} = v^4(v^{3x+1})^{x-3}$
      \item[(f)] $6 + 2\lg{x} = 2$
      \item[(g)] $\lg{\sqrt[3]{4x}} = \frac{1}{2}$
    \end{enumerate}
  \end{multicols}
\end{exercise}

\vspace{0.1cm}

\begin{exercise}
  Suchen Sie nach Nullstellen folgender Polynomfunktionen, indem Sie sie (wenn m\"oglich) in Linearfaktoren zerlegen:\\
  Frage: Ist das f\"ur jedes Polynom m\"oglich?\\
  Hinweis: Nullstellen m\"ussen Faktoren des konstanten Gliedes sein.
  \begin{enumerate}
    \item[(a)] $f(x) = x^3-2x^2-5x+6$
    \item[(b)] $f(x) = 2x^4+2x^3+2x^2+2x$ 
  \end{enumerate}
\end{exercise}

\begin{exercise}
  L\"osen Sie die folgenden Gleichungen und pr\"ufen Sie, f\"ur welche $x$ sie definiert sind:
  \begin{enumerate}
    \item[(a)] $\frac{4}{x^2-4}+\frac{1}{x-2} = \frac{2x-4}{2(x+2)}$ 
    \item[(b)] $\sqrt{\frac{4}{9}x^2+\frac{4}{3}x+1}+2x=7$
  \end{enumerate}
\end{exercise} 

\vspace{2cm}

\begin{exercise}
  L\"osen Sie die Gleichungen mit dem Substutionsverfahren und pr\"ufen Sie die L\"osungen:
  \begin{enumerate}
    \item[(a)] $x^6-16x^3=-64$
    \item[(b)] $e^x+e^{-x}=2$
    \item[(c)] $(\ln{x})^2-9\ln{x}=-20$
  \end{enumerate}
\end{exercise}

\vspace{0.1cm}

\begin{exercise}
  Bestimmen Sie jeweils den Definitionsbereich von $f(x)$, die Umkehrfunktion $f^{-1}(x)$ und den Definitionsbereich von $f^{-1}(x)$:
  \begin{enumerate}
    \item[(a)] $f(x)=6x+3$
    \item[(b)] $f(x)=\frac{1}{25}x^2$
    \item[(c)] $f(x)=\frac{x-1}{x+4}$
  \end{enumerate}
\end{exercise}

\vspace{0.1cm}

\begin{exercise}
  Bestimmen Sie die L\"osungsmengen der Ungleichungen und der Gleichung:
  \begin{enumerate}
    \item[(a)] $x+2<5x-8$
    \item[(b)] $(x-3)^2<4$
    \item[(c)] $x^2+3\leq 2$
    \item[(d)] $\vert \frac{x}{5}-2 \vert \leq 3$
    \item[(e)] $\vert x^2-2x \vert = 1$
  \end{enumerate}
\end{exercise}



\end{document}
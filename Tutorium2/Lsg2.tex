\documentclass[12pt]{article}

\usepackage{amsmath,amsthm,amsxtra,amssymb}
\usepackage{multicol}
\usepackage{rotating}
\usepackage{moreverb}
\usepackage{epsfig}

\newtheorem{satz}{Satz}
\newtheorem{thm}[satz]{Theorem}
\newtheorem{exercise}[satz]{Aufgabe}

\begin{document}

   \pagestyle{empty}
   \parindent 0cm
   \begin{minipage}{14cm}
   \footnotesize{\textbf{HOCHSCHULE} \hfill Dipl.-Math. Xenia Bogomolec\\
  \textbf{HANNOVER}\\
   UNIVERSITY OF\\
   APPLIED SCIENCES\\
   AND ARTS
   }
   \end{minipage}
   \vspace{1.0cm}

   \begin{center}
     {\Large \bf Br\"uckenkurs Mathematik} \\
     \vspace{0.5cm}
     {\large L\"osungen zum \"Ubungsblatt 2}  \\

  \end{center}
   \vspace{0.5cm}
   \normalsize
   \parindent0cm
   
  \begin{exercise}
  Vereinfachen Sie folgende Potenzen ohne Taschenrechner:
  \begin{multicols}{2}
  \begin{enumerate}
  \item[(a)] $3000^4 = 81\cdot10^{12}$
  \item[(b)] $(-2x)^3(-0,5y)^3 = x^3y^3$
  \item[(c)] $5(ac)^6 \cdot 12c^3 \cdot a^3 = 60a^9b^6c^3$
  \item[(d)] $(\frac{1}{5})^4 \cdot 5^{-3} \cdot (-5)^6 = \frac{1}{5}$
  \end{enumerate}
  \end{multicols}
   \end{exercise}

   \vspace{0.1cm}

  \begin{exercise}
  Fassen Sie folgende Summen geschickt zusammen:
  \begin{enumerate}
  \item[(a)] $4a^3+6a-2a^2+2a^3+8a^2 = 6a(a^2+a+1)$
  \item[(b)] $(\frac{1}{x})^{-8}+(x^2)^4+3x^8 = 5x^8$
  \item[(d)] $(-3)^3+4(-6)^2+(\frac{1}{3})^{-3} = 144$ 
  \end{enumerate}
   \end{exercise}

   \vspace{0.1cm}
   
   \begin{exercise}
  Vereinfachen Sie die Quotienten:
  \begin{multicols}{2}
  \begin{enumerate}
  \item[(a)] $(\frac{a^8}{a^{-9}})^{-1} = \frac{1}{a^{17}}$ 
  \item[(b)] $\frac{15x^9y^{11}}{3x^4y^5} = 5x^5y^6$
  \item[(c)] $\frac{c^7x\ln^3{x}}{3(c\ln{x})^8} = \frac{1}{3c(\ln{x})^5}$
  \end{enumerate}
  \end{multicols}
   \end{exercise} 

   \vspace{0.1cm}

   \begin{exercise}
  Vereinfachen Sie die Wurzeln:
  \begin{multicols}{2}
  \begin{enumerate}
  \item[(a)] $\sqrt{x^{12}} = x^6$
  \item[(b)] $\sqrt[3]{x^9(x+2)^4} = x^3(x+2)^{\frac{4}{3}}$
  \item[(c)] $\sqrt[4]{\sqrt{x^3}} = x^{\frac{3}{8}}$
  \item[(d)] $\sqrt[3]{a}\cdot\sqrt[7]{a^2} = \sqrt[21]{a^{13}}$
  \item[(e)] $\frac{(x+1)^4\cdot\sqrt{(x^2+4)x^3}}{\sqrt{(x+1)^3}\cdot x^2 \cdot (x^2+4)^2} 
  = 
  = \frac{\sqrt{(x+1)^5}}{\sqrt{(x^2+4)^3}\cdot\sqrt{x}}$
  \end{enumerate}
  \end{multicols}
   \end{exercise}

   \vspace{0.1cm}

   \begin{exercise}
  K\"urzen Sie die folgenden Br\"uche:
  \begin{multicols}{2}
  \begin{enumerate}
  \item[(a)] $\frac{a^6c^2+a^{13}c^5}{a^6c^2} = 1 + a^7c^3$
  \item[(b)] $\frac{z^{n-5}-z^{n+3}}{z^n} = z^{-5} - z^3$
  \item[(c)] $\frac{x^{m-4}}{x^{m+2}-x^2} = \frac{x^{m-6}}{x^m -1}$
  \end{enumerate}
  \end{multicols}
   \end{exercise}

   \vspace{0.1cm}

   \begin{exercise}
  Berechnen Sie $x$ (bei b) und c) ist ein Taschenrechner erlaubt):
  \begin{enumerate}
  \item[(a)] $x = \lg{1000} \Leftrightarrow x = 3$
  \item[(b)] $x = \log_7{33} \Leftrightarrow x = \frac{\lg{33}}{\lg{7}}\Leftrightarrow x \approx 1,7568$
  \item[(c)] $2^x = 14 \Leftrightarrow x = \frac{\lg{14}}{\lg{2}}\Leftrightarrow x \approx 3,8074$
  \item[(d)] $2^{x+1} = 16 \Leftrightarrow x = 3$
  \item[(e)] $e^x = 1 \Leftrightarrow x = 0$
  \item[(f)] $3^x = 27^2 \Leftrightarrow x = 6$
  \item[(g)] $5^x = \frac{1}{\sqrt[3]{5}} \Leftrightarrow x = -\frac{1}{3}$
  \end{enumerate}
   \end{exercise}

   \vspace{0.1cm}

   \begin{exercise}
  Berechnen Sie die Terme:
  \begin{enumerate}
  \item[(a)] $\lg{(100)^5} = 5\lg{(10^2)} = 5\cdot 2\lg{(10)} = 10$
  \item[(b)] $2\log_{12}{3}+4\log_{12}{2} = \log_{12}{(3^2\cdot2^4)} = \log_{12}{(3\cdot2^2)^2} = log_{12}{(12)^2} = 2$
  \end{enumerate}
   \end{exercise}

\vspace{0.1cm}

    \begin{exercise}
  Setzen Sie $<,>$ oder $=$ ein, damit folgende Aussagen wahr sind:
  \begin{enumerate}
  \item[(a)] \text{Wenn} $p<q$ \textrm{und} $a = 1$, \textrm{dann gilt} $a^p = a^q.$
  \item[(b)] \textrm{Wenn} $p>q$ \textrm{und} $c = 0$, \textrm{dann gilt} $c^p = c^q.$
  \item[(c)] \textrm{Wenn} $p<q$ \textrm{und} $x > 1$, \textrm{dann gilt} $x^p < x^q.$
  \item[(c)] \textrm{Wenn} $p>q$ \textrm{und} $0<c<1$, \textrm{dann gilt} $c^p < c^q.$
  \end{enumerate}
   \end{exercise}


\end{document}
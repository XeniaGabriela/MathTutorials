\documentclass[12pt]{article}

\usepackage{amsmath,amsthm,amsxtra,amssymb}
\usepackage{multicol}
\usepackage{rotating}
\usepackage{moreverb}
\usepackage{epsfig}

\newtheorem{satz}{Satz}
\newtheorem{thm}[satz]{Theorem}
\newtheorem{exercise}[satz]{Aufgabe}

\begin{document}

  \pagestyle{empty}
  \parindent 0cm
  \begin{minipage}{14cm}
  \footnotesize{\textbf{HOCHSCHULE} \hfill Dipl.-Math. Xenia Bogomolec\\
  \textbf{HANNOVER}\\
   UNIVERSITY OF\\
   APPLIED SCIENCES\\
   AND ARTS
   }
  \end{minipage}
  \vspace{1.0cm}

  \begin{center}
     {\Large \bf Br\"uckenkurs Mathematik} \\
     \vspace{0.5cm}
     {\large L\"osungen des Anf\"anger-\"Ubungsblatts zur Differential- und Integralrechnung}
  \end{center}
  \normalsize
  \parindent0cm

  \vspace{1.0cm}
  
  \large{Differenzieren Sie $f(x)$:}
   
  \vspace{0.5cm}
   
  \begin{exercise}\hfill
  \begin{enumerate}
  \item[(a)] $f(x)=x^2 \Rightarrow f^\prime(x)=2x$
  \item[(b)] $f(x)=3x^2 \Rightarrow f^\prime(x)=6x$
  \item[(c)] $f(x)=2x^4 \Rightarrow f^\prime(x)=8x^3$
  \item[(d)] $f(x)=x^2+2x^4 \Rightarrow f^\prime(x)=2x+8x^3$ (Summenregel)
  \end{enumerate}
  \end{exercise}

  \vspace{0.3cm}

  \begin{exercise}\hfill
  \begin{enumerate}
  \item[(a)] $f(x)=\ln{x} \Rightarrow f^\prime(x)=\frac{1}{x}$
  \item[(b)] $f(x)=5 \ln{x} \Rightarrow f^\prime(x)=\frac{5}{x}$
  \item[(c)] $f(x)=e^x \Rightarrow f^\prime(x)=e^x$
  \item[(d)] $f(x)=e^{2x} \Rightarrow f^\prime(x)=2e^{2x}$ (Kettenregel) 
  \end{enumerate}
  \end{exercise}

  \vspace{0.3cm}
   
  \begin{exercise}\hfill
  \begin{enumerate}
  \item[(a)] $f(x)=\sin{x} \Rightarrow f^\prime(x)=\cos{x}$ 
  \item[(b)] $f(x)=\cos{x} \Rightarrow f^\prime(x)=-\sin{x}$
  \item[(c)] $f(x)=\sin{x} \cdot \cos{x} \Rightarrow f^\prime(x)=\cos^2{x}-\sin^2{x}$ (Produktregel)
  \item[(d)] $f(x)=\tan{x} = \frac{\sin{x}}{\cos{x}} \Rightarrow f^\prime(x)=\frac{\cos^2{x}+\sin^2{x}}{\cos^2{x}} 
  = \frac{1}{\cos^2{x}}$ (Quotientenregel, Pythagoras)
  \end{enumerate}
  \end{exercise} 

  \vspace{0.7cm}

  \large{Integrieren Sie $f(x)$:}

  \vspace{0.5cm}

  \begin{exercise}\hfill
  \begin{enumerate}
  \item[(a)] $f(x)=x^3 \Rightarrow F(x)=\frac{1}{4}x^4+C$ 
  \item[(b)] $f(x)=2x^5 \Rightarrow F(x)=\frac{1}{3}x^6+C$
  \item[(c)] $f(x)=x^3+2x^5 \Rightarrow F(x)=\frac{1}{4}x^4+\frac{1}{3}x^6+C$
  \end{enumerate}
  \end{exercise}

  \vspace{0.3cm}

  \begin{exercise}\hfill
  \begin{enumerate}
  \item[(a)] $f(x)=\sin{x} \Rightarrow F(x)=-\cos{x}+C$
  \item[(b)] $f(x)=\cos{x} \Rightarrow F(x)=\sin{x}+C$
  \item[(c)] $f(x)=\sin^2{x}-\cos^2{x} \Rightarrow F(x)=-\sin{x} \cdot \cos{x}$ \\(Vgl. Aufgabe 3d))
  \end{enumerate}
  \end{exercise}

  \vspace{0.3cm}

  \begin{exercise}\hfill
  \begin{enumerate}
  \item[(a)] $f(x)=\frac{4}{x} \Rightarrow F(x)=4\ln{x}+C$
  \item[(b)] $f(x)=e^{2x} \Rightarrow F(x)=\frac{1}{2}e^{2x}+C$
  \end{enumerate}
  \end{exercise}


\end{document}
\documentclass[12pt]{article}

\usepackage{amsmath,amsthm,amsxtra,amssymb}
\usepackage{multicol}
\usepackage{rotating}
\usepackage{moreverb}
\usepackage{epsfig}

\newtheorem{satz}{Satz}
\newtheorem{thm}[satz]{Theorem}
\newtheorem{exercise}[satz]{Aufgabe}

\begin{document}

\pagestyle{empty}
\parindent 0cm
\begin{minipage}{14cm}
  \footnotesize{\textbf{HOCHSCHULE} \hfill Dipl.-Math. Xenia Bogomolec\\
  \textbf{HANNOVER}\\
    UNIVERSITY OF\\
    APPLIED SCIENCES\\
    AND ARTS
    }
\end{minipage}
\vspace{1.0cm}

\begin{center}
  {\Large \bf Br\"uckenkurs Mathematik} \\
  \vspace{0.5cm}
  {\large L\"osungen zum \"Ubungsblatt 4}  \\
\end{center}

\vspace{0.5cm}
\normalsize
\parindent0cm

\begin{exercise}
  Zeichnen Sie den Graphen der Funktion $f(x) = x^2-2x+3$ im Intervall $[-3,4]$. \\
  L\"osung: Die Punkte ergeben miteinander verbunden eine Parabel\\
  $f(-3) = 18, f(-2) = 11, f(-1) = 6, f(0) = 3, f(1) = 2$ (Minimum), $f(2) = 3, f(3) = 6, f(4) = 11$ 
\end{exercise}

\vspace{0.1cm}

\begin{exercise}
  Bestimmen Sie jeweils die Definitionsmenge, die Bildmenge, alle Nullstellen, den Scheitelpunkt und die Umkehrfunktion:
  \begin{enumerate}
    \item[(a)] $f(x) = 3x^2+5:$\\
      $D_f = \{x \in \mathbb{R}\}$\\
      $Im f = \{y \in \mathbb{R}\mid y\geq 5\}$\\
      $f$ hat keine Nullstellen!\\
      Scheitelpunkt ist $(0,5),$ denn $x^2$ ist immer gr\"osser als $0$ f\"ur $x \neq 0,$ also muss bei $x = 0 $ das Minimum von $f$ sein.\\
      $f^{-1}(x) = \sqrt{\frac{x-5}{3}}$
    \item[(b)] $f(x) = \frac{1}{x+4}:$\\
      $D_f = \{x \in \mathbb{R} \mid x \neq -4\}$\\
      $Im f = \{y \in \mathbb{R}\mid y\neq 0\} \Rightarrow$ es gibt keine Nullstellen!\\
      Einen Scheitelpunkt gibt es nicht, aber wir haben einen Pol bei $x = -4$.\\
      $f^{-1}(x) = \frac{1}{x} - 4$ 
  \end{enumerate}
\end{exercise}

\vspace{0.1cm}

\begin{exercise}
  Zeichnen Sie die folgenden Betragsfunktionen:
  \begin{enumerate}
    \item[(a)] $f(x) = \vert 2x^2 \vert$ ist eine um Faktor $2$ gestreckte Parabel. 
    \item[(b)] $f(x) = \vert x^2 - 9 \vert$ ist eine um $9$ Einheiten nach unten versetzte Parabel, zwischen Nullstellen $-3$ und $3$ an der $x$-Achse nach oben gespiegelt.
    \item[(c)]$f(x) = \vert x \vert$ entspricht der oberen H\"alfte der Achsendiagonalen.
  \end{enumerate}
\end{exercise} 

\vspace{0.1cm}

\begin{exercise}
  F\"uhren Sie die Polynom-Divisionen durch:
  \begin{enumerate}
  \item[(a)] $(x^3+7x^2+9x-5)\div(x+5) = (x^2+2x-1)$
  \item[(b)] $(x^5-x^4-13x^3+16x^2+13x-10)\div(x^2+3x-2) = (x^3-4x^2+x+5)$
  \item[(c)] $(x^3+3x^2+3x+1)\div(x+1) = (x^2+2x+1)$
  \end{enumerate}
\end{exercise}

\vspace{0.1cm}

\begin{exercise}
  Beschreiben Sie Symmetrie, Monotonieverhalten und Achsenschnittpunkte der folgenden Graphen:
  \begin{enumerate}
    \item[(a)] $f(x) = x^6 + 14$ ist eine gerade Funktion, also achsensysmmetrisch bezgl. der y-Achse. Der Graph entspricht der Parabel $x^6$ um 14 Einheiten in $y$-Richtung nach oben verschoben. Somit gibt es keinen Schnittpunkt mit der $x$-Achse, aber der Schnittpunkt mit der mit der $y$-Achse ist bei $(0,14)$.
    \item[(b)] $f(x) = 3x^{-4}$ ist auch eine gerade Funktion, symmetrisch bezgl. der $y$-Achse. Im ersten Quadranten der Gau\ss schen Zahlenebene \"ahnelt sie der positiven H\"alfte einer Hyperbel. $f$ hat gar keine Achsenschnittpunkte.
    \item[(c)] $f(x) = 2(x-2)^3 + 1$ entspricht der Funktion $g(x) = x^3$ in $x$-Richtung um 2 und in $y$-Richtung um 1 Einheit verschoben. Zudem ist sie um Faktor 3 in $y$-Richtung gestreckt. Der Schnittpunkt mit der $y$-Achse ist $(0,-15)$ und Schnittpunkt mit der $x$-Achse ist $(2-\frac{1}{\sqrt[3]{2}},0)$. (Jede Polynomfunktion mit ungeradem gr\"o\ss ten Koeffizienten hat einen Nullpunkt!\\
  Der Beweis auf youtube: https://www.youtube.com/watch?v=8l-La9HEUIU)
  \end{enumerate}
\end{exercise}

\vspace{0.1cm}

\begin{exercise}
  Rechnen Sie von Grad ins Bogenma\ss \, um oder umgekehrt:
  \begin{multicols}{2}
    \begin{enumerate}
    \item[(a)] $30^\circ = \frac{\pi}{6}$
    \item[(b)] $-45^\circ = -\frac{\pi}{4} = \frac{7\pi}{4}$
    \item[(c)] $135^\circ =\frac{3\pi}{4}$
    \item[(d)] $\frac{\pi}{4} = 45^\circ$
    \item[(e)] $-\frac{5\pi}{6} = -150^\circ = 210^\circ$
    \item[(f)] $\frac{\pi}{3} = 60^\circ$
    \end{enumerate}
  \end{multicols}
\end{exercise}

\vspace{0.1cm}

\begin{exercise}
  Gegeben seien rechtwinklige Dreiecke mit Katheten $a$ und $b$ und Hypotenuse $c$ und Winkeln $\alpha$(gegen\"uber $a$), $\beta$(gegen\"uber $b$) und $\gamma = 90^\circ$. Berechnen Sie die fehlenden Seiten oder Winkel:
  \begin{enumerate}
    \item[(a)] $a = 3cm, b = 4cm:$\\
    $c=\sqrt{a^2+b^2} = \sqrt{9+16} = 5, \,\alpha = \sin{\frac{3}{5}}, \,\beta = \sin{\frac{4}{5}}$
    \item[(b)] $c = 10cm, \alpha = 45^\circ:$\\
    $\beta = \alpha = 45^\circ,$ und wegen $a = b $ gilt $2a^2 = c^2 \Rightarrow a = b = \sqrt{\frac{c^2}{2}} = \sqrt{\frac{100}{2}} = \sqrt{50} = \sqrt{2} \cdot 5$ 
  \end{enumerate}
\end{exercise}

\vspace{0.1cm}

\begin{exercise}
  Bestimmen Sie erst Amplitude, Periode und Phasenverschiebung der Schwingungsfunktion $f(x) = 3 \sin{(2x-\frac{\pi}{4})}$ und zeichnen Sie nachher ihren Verlauf.\\
  $f(x) = 3 \sin{(2x-\frac{\pi}{4}})= A \sin{(bx+c)}$\\
  Amplitude $A=3$, Periode $P = \frac{2\pi}{b} = \frac{2\pi}{2} = \pi$, Phasenverschiebung $x_0 = \frac{c}{b} = -\frac{\pi}{8}$
\end{exercise}



\end{document}
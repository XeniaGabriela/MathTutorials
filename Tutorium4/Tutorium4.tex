\documentclass[12pt]{article}

\usepackage{amsmath,amsthm,amsxtra,amssymb}
\usepackage{multicol}
\usepackage{rotating}
\usepackage{moreverb}
\usepackage{epsfig}

\newtheorem{satz}{Satz}
\newtheorem{thm}[satz]{Theorem}
\newtheorem{exercise}[satz]{Aufgabe}

\begin{document}

\pagestyle{empty}
\parindent 0cm
\begin{minipage}{14cm}
\footnotesize{\textbf{HOCHSCHULE} \hfill Dipl.-Math. Xenia Bogomolec\\
\textbf{HANNOVER}\\
  UNIVERSITY OF\\
  APPLIED SCIENCES\\
  AND ARTS
  }
\end{minipage}
\vspace{1.0cm}

\begin{center}
   {\Large \bf Br\"uckenkurs Mathematik} \\
   \vspace{0.5cm}
   {\large \"Ubungsblatt 4 (Elementare Funktionen, Trigonometrie)}  \\
\end{center}

\vspace{0.5cm}
\normalsize
\parindent0cm

\begin{exercise}
    Zeichnen Sie den Graphen der Funktion $f(x) = x^2-2x+3$ im Intervall $[-3,4]$.
\end{exercise}

\vspace{0.1cm}

\begin{exercise}
  Bestimmen Sie jeweils die Definitionsmenge, die Bildmenge, alle Nullstellen, den Scheitelpunkt und die Umkehrfunktion:
  \begin{enumerate}
    \item[(a)] $f(x) = 3x^2+5$
    \item[(b)] $f(x) = \frac{1}{x+4}$ 
  \end{enumerate}
\end{exercise}

\vspace{0.1cm}

\begin{exercise}
  Zeichnen Sie die folgenden Betragsfunktionen:
  \begin{enumerate}
    \item[(a)] $f(x) = \vert 2x^2 \vert$ 
    \item[(b)] $f(x) = \vert x^2 - 9 \vert$
    \item[(c)]$f(x) = \vert x \vert$
  \end{enumerate}
\end{exercise} 

\vspace{0.1cm}

\begin{exercise}
  F\"uhren Sie die Polynom-Divisionen durch:
  \begin{enumerate}
    \item[(a)] $(x^3+7x^2+9x-5)\div(x+5)$
    \item[(b)] $(x^5-x^4-13x^3+16x^2+13x-10)\div(x^2+3x-2)$
    \item[(c)] $(x^3+3x^2+3x+1)\div(x+1)$
  \end{enumerate}
\end{exercise}

\vspace{1.0cm}

\begin{exercise}
  Beschreiben Sie Symmetrie, Monotonieverhalten und Achsenschnittpunkte der folgenden Graphen:
  \begin{enumerate}
    \item[(a)] $f(x) = x^6 + 14$
    \item[(b)] $f(x) = 3x^{-4}$
    \item[(c)] $f(x) = 2(x-2)^3 + 1$
  \end{enumerate}
\end{exercise}

\vspace{0.1cm}

\begin{exercise}
Rechnen Sie von Grad ins Bogenma\ss \, um oder umgekehrt:
  \begin{multicols}{2}
    \begin{enumerate}
      \item[(a)] $30^\circ$
      \item[(b)] $-45^\circ$
      \item[(c)] $135^\circ$
      \item[(d)] $\frac{\pi}{4}$
      \item[(e)] $-\frac{5\pi}{6}$
      \item[(f)] $\frac{\pi}{3}$
    \end{enumerate}
  \end{multicols}
\end{exercise}

\vspace{0.1cm}

\begin{exercise}
  Gegeben seien rechtwinklige Dreiecke mit Katheten $a$ und $b$ und Hypotenuse $c$ und Winkeln $\alpha$(gegen\"uber $a$),$\beta$ (gegen\"uber $b$) und $\gamma = 90^\circ$. Berechnen Sie die fehlenden Seiten oder Winkel:
  \begin{enumerate}
    \item[(a)] $a = 3cm, b = 4cm$
    \item[(b)] $c = 10cm, \alpha = 45^\circ$
  \end{enumerate}
\end{exercise}

\vspace{0.1cm}

\begin{exercise}
  Bestimmen Sie erst Amplitude, Periode und Phasenverschiebung der Schwingungsfunktion $f(x) = 3 \sin{(2x-\frac{\pi}{4})}$ und zeichnen Sie nachher ihren Verlauf.
\end{exercise}



\end{document}
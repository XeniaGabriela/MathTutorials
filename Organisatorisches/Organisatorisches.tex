\documentclass[12pt]{article}

\usepackage{amsmath,amsthm,amsxtra,amssymb}
\usepackage{multicol}
\usepackage{rotating}
\usepackage{moreverb}
\usepackage{epsfig}
\usepackage{enumitem}
\setlist[enumerate]{itemsep=0mm}

\newtheorem{satz}{Satz}
\newtheorem{thm}[satz]{Theorem}
\newtheorem{exercise}[satz]{Aufgabe}

\begin{document}

\pagestyle{empty}
\parindent 0cm
\begin{minipage}{14cm}
  \footnotesize{
  \textbf{HOCHSCHULE} \hfill Dipl.-Math. Xenia Bogomolec\\
  \textbf{HANNOVER}   \hfill indigomind@gmx.de  \\
  UNIVERSITY OF\\
  APPLIED SCIENCES\\
  AND ARTS
  }
\end{minipage}
\vspace{0.5cm}

{\large \bf Organisatorisches zum Br\"uckenkurs Mathematik}

\vspace{0.5cm}
\normalsize
\parindent0cm

Der Kurs findet vom 10. bis zum 17.9. vormittags zwischen 9h und 12:30h statt. Von 10:30h bis 11h ist die erste Pause. Bitte bringen Sie Papier und Stifte zum Schreiben mit. Ich werde Einiges pr\"asentieren, das noch nicht im Skript festgehalten ist. \\ \\
Die Tutorien f\"ur angehende Ingenieure finden nachmittags unter der Leitung erfahrener Studenten statt. Die Gruppen werden am 17.9. nach der ersten Kurs-Lesung eingeteilt. Die leitenden Studenten werden sich individuell entscheiden, ob sie die Tutorien um 13h oder um 13:30h beginnen wollen. Pro Nachmittag sind 4 Unterrichts-Einheiten geplant.\\
Frau Bogomolec wird die Tutorien f\"ur Mathematik-Studenten leiten. Sie finden jeweils von 13:30h - 17:00h statt. \\ \\ 
Neben dem Bearbeiten von \"Ubungsaufgaben werden wir Sie in der Zeit mit der Infrastruktur der Hochschule vertraut machen.\\ \\


Die \"Ubungsbl\"atter zu den begleitenden Tutorien sind auf \\
http://f1.hs-hannover.de/service/brueckenkurs/index.html zum Download be\-reit\-gestellt. \\

{\bf Tutorien f\"ur angehende Ingenieure:} 

\begin{enumerate}
  \item[] \"Ubungsblatt 1: Grundlagen
  \item[] \"Ubungsblatt 2: Potenzen, Wurzeln, Logarithmen
  \item[] \"Ubungsblatt 3: Vektoren
  \item[] \"Ubungsblatt 4: Elementare Funktionen, Trigonometire
  \item[] \"Ubungsblatt 5: Gleichungen und Ungleichungen
  \item[] \"Ubungsblatt 6: Differentialrechnung und Extras
\end{enumerate}

{\bf Tutorien f\"ur Mathematik-Studenten:} 

\begin{enumerate}
  \item[] \"Ubungsblatt 2: Potenzen, Wurzeln, Logarithmen
  \item[] \"Ubungsblatt 3: Vektoren
  \item[] \"Ubungsblatt 4: Elementare Funktionen, Trigonometire
  \item[] \"Ubungsblatt 5: Gleichungen und Ungleichungen
  \item[] \"Ubungsblatt 6: Differentialrechnung und Extras
  \item[] \"Ubungsblatt 7: Lineare Algebra und Kurvendiskussionen
\end{enumerate}

\vspace{0.5cm}

Studenten, die sich noch nicht oder nur wenig mit Differential- und Integral-Rechnung auseinandergesetzt haben, empfehlen wir das Extra-\"Ubungsblatt Anf\"angerDiffInt.pdf zu bearbeiten. Dieses \"Ubungsblatt wird nicht in den Tutorien besprochen. Man darf jedoch jederzeit mit Fragen auf Frau Bogomolec oder die Tutoren zugehen. \\



\end{document}
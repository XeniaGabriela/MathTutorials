\documentclass[12pt]{article}

\usepackage{amsmath,amsthm,amsxtra,amssymb}
\usepackage{multicol}
\usepackage{rotating}
\usepackage{moreverb}
\usepackage{epsfig}

\newtheorem{satz}{Satz}
\newtheorem{thm}[satz]{Theorem}
\newtheorem{exercise}[satz]{Aufgabe}

\begin{document}

  \pagestyle{empty}
  \parindent 0cm
  \begin{minipage}{14cm}
  \footnotesize{\textbf{HOCHSCHULE} \hfill Dipl.-Math. Xenia Bogomolec\\
  \textbf{HANNOVER}\\
   UNIVERSITY OF\\
   APPLIED SCIENCES\\
   AND ARTS
   }
  \end{minipage}
  \vspace{1.0cm}

  \begin{center}
     {\Large \bf Br\"uckenkurs Mathematik} \\
     \vspace{0.5cm}
     {\large Anf\"anger-\"Ubungsblatt zur Differential- und Integralrechnung}
  \end{center}
  \normalsize
  \parindent0cm

  \vspace{1.0cm}
  
  \large{Differenzieren Sie $f(x)$:}
   
  \vspace{0.5cm}
   
  \begin{exercise}\hfill
  \begin{multicols}{2}
  \begin{enumerate}
  \item[(a)] $f(x)=x^2$
  \item[(b)] $f(x)=3x^2$
  \item[(c)] $f(x)=2x^4$
  \item[(d)] $f(x)=x^2+2x^4$
  \end{enumerate}
  \end{multicols}
  \end{exercise}

  \vspace{0.3cm}

  \begin{exercise}\hfill
  \begin{multicols}{2}
  \begin{enumerate}
  \item[(a)] $f(x)=\ln{x}$
  \item[(b)] $f(x)=5 \ln{x}$
  \item[(c)] $f(x)=e^x$
  \item[(d)] $f(x)=e^{2x}$  
  \end{enumerate}
  \end{multicols}
  \end{exercise}

  \vspace{0.3cm}
   
  \begin{exercise}\hfill
  \begin{multicols}{2}
  \begin{enumerate}
  \item[(a)] $f(x)=\sin{x}$ 
  \item[(b)] $f(x)=\cos{x}$
  \item[(c)] $f(x)=\sin{x} \cdot \cos{x}$
  \item[(d)] $f(x)=\tan{x}$ 
  \end{enumerate}
  \end{multicols}
  \end{exercise} 

  \vspace{2.0cm}

  \large{Integrieren Sie $f(x)$:}

  \vspace{0.5cm}

  \begin{exercise}\hfill
  \begin{enumerate}
  \item[(a)] $f(x)=x^3$ 
  \item[(b)] $f(x)=2x^5$
  \item[(c)] $f(x)=x^3+2x^5$
  \end{enumerate}
  \end{exercise}

  \vspace{0.3cm}

  \begin{exercise}\hfill
  \begin{enumerate}
  \item[(a)] $f(x)=\sin{x}$
  \item[(b)] $f(x)=\cos{x}$
  \item[(c)] $f(x)=\sin^2{x}-\cos^2{x}$
  \end{enumerate}
  \end{exercise}

  \vspace{0.3cm}

  \begin{exercise}\hfill
  \begin{enumerate}
  \item[(a)] $f(x)=\frac{4}{x}$
  \item[(b)] $f(x)=e^{2x}$
  \end{enumerate}
  \end{exercise}


\end{document}
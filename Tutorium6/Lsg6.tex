\documentclass[12pt]{article}

\usepackage{amsmath,amsthm,amsxtra,amssymb}
\usepackage{multicol}
\usepackage{rotating}
\usepackage{moreverb}
\usepackage{epsfig}

\newtheorem{satz}{Satz}
\newtheorem{thm}[satz]{Theorem}
\newtheorem{exercise}[satz]{Aufgabe}

\begin{document}

   \pagestyle{empty}
   \parindent 0cm
   \begin{minipage}{14cm}
   \footnotesize{\textbf{HOCHSCHULE} \hfill Dipl.-Math. Xenia Bogomolec\\
  \textbf{HANNOVER}\\
   UNIVERSITY OF\\
   APPLIED SCIENCES\\
   AND ARTS
   }
   \end{minipage}
   \vspace{1.0cm}

   \begin{center}
     {\Large \bf Br\"uckenkurs Mathematik zum Wintersemester 2015/2016} \\
     \vspace{0.5cm}
     {\large L\"osungen zum \"Ubungsblatt 6}  \\

  \end{center}
   \vspace{0.5cm}
   \normalsize
   \parindent0cm

   
  \begin{exercise}
  Differenzieren Sie $f(x)$ nach der Summenregel:
  \begin{enumerate}
  \item[(a)] $f(x)=\frac{3}{4}x^6+\frac{1}{2}x^3-5x+8\Rightarrow
             f^\prime(x)=\frac{9}{2}x^5+\frac{3}{2}x^2-5$
  \item[(b)] $f(x)=ax^4-2bx^3+cx^2-4dx\Rightarrow
             f^\prime(x)=4ax^3-6bx^2+2cx-4d$
  \item[(c)] $f(x)=a\sin{x}+b\cos{x}+cx\Rightarrow
             f^\prime(x)=a\cos{x}-b\sin{x}+c$
  \item[(d)] $f(x)=2\sqrt{x^5}-5\sqrt[4]{x}=2x^{\frac{5}{2}}-5x^{\frac{1}{4}}\Rightarrow
             f^\prime(x)=5x^{\frac{3}{2}}-\frac{5}{4}x^{-\frac{3}{4}}=5\sqrt{x^3}-\frac{5}{4\sqrt[4]{x^3}}$
  \item[(e)] $f(x)=x^{-3}-x^{-7}\Rightarrow
             f^\prime(x)=-3x^{-4}+7x^{-8}$
  \item[(f)] $f(x)=e^x+e^{3x}-\ln{x}\Rightarrow
             f^\prime(x)=e^x+3e^{3x}-\frac{1}{x}$
  \end{enumerate}
   \end{exercise}

  \begin{exercise}
  Differenzieren Sie $f(x)$ nach der Produktregel:
  \begin{enumerate}
  \item[(a)] $f(x)=\sin{x}\cdot\cos{x}\Rightarrow f^\prime(x)=\cos^2{x}-\sin^2{x}$
  \item[(b)] $f(x)=x^3\cdot\ln{x}\Rightarrow f^\prime(x)=x^2(1+3\ln{x})$
  \item[(c)] $f(x)=(4x^3-2x+1)\cdot(x^2-2x+5)\Rightarrow f^\prime(x)=20x^4-32x^3+54x^2+10x-12$
  \item[(d)] $f(x)=e^{2x}\cdot\sin{x}\Rightarrow f^\prime(x)=e^{2x}(2\sin{x}+\cos{x})$  
  \end{enumerate}
   \end{exercise}
   
   \begin{exercise}
  Differenzieren Sie $f(x)$ nach der Quotientenregel:
  \begin{enumerate}
  \item[(a)] $f(x)=\frac{x}{x+1} \Rightarrow f^\prime(x)=\frac{1}{(x+1)^2}$ 
  \item[(b)] $f(x)=\frac{\ln{x}}{x^4} \Rightarrow f^\prime(x)=\frac{1-4\ln{x}}{x^5}$
  \item[(c)] $f(x)=\frac{\cos{x}}{e^{2x}} \Rightarrow f^\prime(x)=\frac{-\sin{x}-2\cos            {x}}{e^{2x}}$
  \end{enumerate}
   \end{exercise} 

 


   \begin{exercise}
  Differenzieren Sie $f(x)$ nach der Kettenregel:
  \begin{enumerate}
  \item[(a)] $f(x)=3(5x^2+2x+3)^4 \Rightarrow f^\prime(x)=12(5x^2+2x+3)^3(10x+2)$ 
  \item[(b)] $f(x)=\sin(3x+12) \Rightarrow f^\prime(x)=3\cos{(3x+12)}$
  \item[(c)] $f(x)=\ln{e^{2x}+x^2} = x^2+2x\Rightarrow f^\prime(x)=2x+2$ \,\,
  Kettenregel war nicht n\"otig!
  \item[(d)] $f(x)=e^{\cos{x}} \Rightarrow f^\prime(x)=-\sin{x}\cdot e^{\cos{x}}$
  \end{enumerate}
   \end{exercise}

   \begin{exercise}
  (Zusatzaufgabe) Differenzieren Sie geschickt:
  \begin{enumerate}
  \item[(a)] $f(x)=e^{\ln{(\sin{x})}} \Rightarrow f^\prime(x)=\cos{x}$
  \item[(b)] $f(x)=\cos^2{(2x+3)} \Rightarrow f^\prime(x)=-4\cos{(2x+3)}\sin{(2x+3)}$
  \item[(c)] $f(x)=\ln{\frac{1}{x^2}}+\ln{\frac{x+4}{x}}= -3\ln{x}+\ln{(x+4)} \Rightarrow f^\prime(x)=\frac{-3}{x}+\frac{1}{x+4}$
  \item[(d)] $f(x)=\ln{(\tan{x})} \Rightarrow f^\prime(x)=\frac{1}{\sin{x}\cdot\cos{x}}$
  \end{enumerate}
   \end{exercise}

   \begin{exercise}
  Bestimmen Sie die Definitionsmenge, die Bildmenge, alle Nullstellen, den Scheitelpunkt und die Umkehrfunktion $f(x) = x^2+4x+3:$ \\ \\
  $D_f = \{x \in \mathbb{R}\}$\\
  $Im f = \{y \in \mathbb{R}\mid y\geq -1\}$ (siehe Scheitelpunkt)\\
  Nullstellen: $f(x) = (x+1)(x+3) \Rightarrow x_1 = -1, x_2 = -3$,\\
  Scheitelpunkt ist $(-2,-1)$, denn $f^\prime(x) = 2x+4 = 0 \Leftrightarrow x = -2$ und $f(-2) = -1$.\\
  Umkehrfunktion: $y = x^2+4x+3 = x^2+4x+4-1 \Leftrightarrow y+1=(x+2)^2 \Leftrightarrow \sqrt{y+1} = x+2 \Leftrightarrow \sqrt{y+1} -2 = x \hfill \Longrightarrow f^{-1}(x) = \sqrt{x+1}-2$
   \end{exercise}

     \begin{exercise}
  Berechnen Sie das Volumen des durch $\vec{u}, \vec{v}$ und $\vec{w}$ aufgespannten Spates f\"ur \, \\ \\
  $\vec{u}=\left(\begin{array}{r} 1 \\ 1 \\ 1 \end{array}\right)$,
  $\vec{v}=\left(\begin{array}{r} 4 \\ 2 \\ 4 \end{array}\right)$ und
  $\vec{w}=\left(\begin{array}{r} 1 \\ 2 \\ 3 \end{array}\right)$.\\ \\ \\

  Spatvolumen:
  \[ V=\vert \left| \begin{array}{ccc}
        1 & 4 & 1 \\
        1 & 2 & 2 \\
        1 & 4 & 3 \end{array} \right|\vert
        = \vert 1\cdot \left| \begin{array}{cc} 2 & 2 \\ 4 & 3 \end{array} \right|
              - 1\cdot \left| \begin{array}{cc} 4 & 1 \\ 4 & 3 \end{array} \right|
              + 1\cdot \left| \begin{array}{cc} 4 & 1 \\ 2 & 2 \end{array} \right|
          \vert
        = \vert (6-8) - (12-4) + (8-2) \vert = \vert -4 \vert = 4
  \]  

   

  \[ \textrm{mit Sarrus:}\,\, V=\vert \left| \begin{array}{ccc}
        1 & 4 & 1 \\
        1 & 2 & 2 \\
        1 & 4 & 3 \end{array} \right|
        \begin{array}{cc}
        1 & 4 \\
        1 & 2 \\
        1 & 4 \end{array}
        \vert
        = \vert 6+8+4-2-8-12 \vert = \vert -4 \vert = 4
  \]  

   \end{exercise}

   

\end{document}
\documentclass[12pt]{article}

\usepackage{amsmath,amsthm,amsxtra,amssymb}
\usepackage{multicol}
\usepackage{rotating}
\usepackage{moreverb}
\usepackage{epsfig}

\newtheorem{satz}{Satz}
\newtheorem{thm}[satz]{Theorem}
\newtheorem{exercise}[satz]{Aufgabe}

\begin{document}

\pagestyle{empty}
\parindent 0cm
\begin{minipage}{14cm}
  \footnotesize{\textbf{HOCHSCHULE} \hfill Dipl.-Math. Xenia Bogomolec\\
  \textbf{HANNOVER}\\
  UNIVERSITY OF\\
  APPLIED SCIENCES\\
  AND ARTS
  }
\end{minipage}
\vspace{1.0cm}

\begin{center}
  {\Large \bf Br\"uckenkurs Mathematik} \\
  \vspace{0.5cm}
  {\large \"Ubungsblatt 6 (Differentialrechnung und Extras)}  \\
\end{center}
\vspace{0.5cm}
\normalsize
\parindent0cm

\begin{exercise}
  Differenzieren Sie $f(x)$ nach der Summenregel:
  \begin{multicols}{2}
    \begin{enumerate}
      \item[(a)] $f(x)=\frac{3}{4}x^6+\frac{1}{2}x^3-5x+8$
      \item[(b)] $f(x)=ax^4-2bx^3+cx^2-4dx$
      \item[(c)] $f(x)=a\sin{x}+b\cos{x}+cx$
      \item[(d)] $f(x)=2\sqrt{x^5}-5\sqrt[4]{x}$
      \item[(e)] $f(x)=x^{-3}-x^{-7}$
      \item[(f)] $f(x)=e^x+e^{3x}-\ln{x}$
    \end{enumerate}
  \end{multicols}
\end{exercise}

\vspace{0.1cm}

\begin{exercise}
  Differenzieren Sie $f(x)$ nach der Produktregel:
  \begin{enumerate}
    \item[(a)] $f(x)=\sin{x}\cdot\cos{x}$
    \item[(b)] $f(x)=x^3\cdot\ln{x}$
    \item[(c)] $f(x)=(4x^3-2x+1)\cdot(x^2-2x+5)$
    \item[(d)] $f(x)=e^{2x}\cdot\sin{x}$  
  \end{enumerate}
\end{exercise}

\vspace{0.1cm}

\begin{exercise}
  Differenzieren Sie $f(x)$ nach der Quotientenregel:
  \begin{enumerate}
    \item[(a)] $f(x)=\frac{x}{x+1}$ 
    \item[(b)] $f(x)=\frac{\ln{x}}{x^4}$
    \item[(c)] $f(x)=\frac{\cos{x}}{e^{2x}}$
  \end{enumerate}
\end{exercise} 

\vspace{0.1cm}

\begin{exercise}
  Differenzieren Sie $f(x)$ nach der Kettenregel:
  \begin{multicols}{2}
    \begin{enumerate}
      \item[(a)] $f(x)=3(5x^2+2x+3)^4$ 
      \item[(b)] $f(x)=\sin(3x+12)$
      \item[(c)] $f(x)=\ln{e^{2x}+x^2}$
      \item[(d)] $f(x)=e^{\cos{x}}$
    \end{enumerate}
  \end{multicols}
\end{exercise}

\vspace{0.1cm}

\begin{exercise}
  (Zusatzaufgabe) Differenzieren Sie geschickt:
  \begin{multicols}{2}
    \begin{enumerate}
      \item[(a)] $f(x)=e^{\ln{(\sin{x})}}$
      \item[(b)] $f(x)=\cos^2{(2x+3)}$
      \item[(c)] $f(x)=\ln{\frac{1}{x^2}}+\ln{\frac{x+4}{x}}$
      \item[(d)] $f(x)=\ln{(\tan{x})}$
    \end{enumerate}
  \end{multicols}
\end{exercise}

\vspace{0.1cm}

\begin{exercise}
  Bestimmen Sie die Definitionsmenge, die Bildmenge, alle Nullstellen, den Scheitelpunkt und die Umkehrfunktion von $f(x) = x^2+4x+3$.
\end{exercise}

\vspace{0.1cm}

\begin{exercise}
  Berechnen Sie Volumen des durch $\vec{u}, \vec{v}$ und $\vec{w}$ aufgespannten Spates f\"ur \,
  $\vec{u}=\left(\begin{array}{r} 1 \\ 1 \\ 1 \end{array}\right)$,
  $\vec{v}=\left(\begin{array}{r} 4 \\ 2 \\ 4 \end{array}\right)$ und
  $\vec{w}=\left(\begin{array}{r} 1 \\ 2 \\ 3 \end{array}\right)$.
\end{exercise}


\end{document}
\documentclass[12pt]{article}

\usepackage{amsmath,amsthm,amsxtra,amssymb}
\usepackage{multicol}
\usepackage{rotating}
\usepackage{moreverb}
\usepackage{epsfig}

\newtheorem{satz}{Satz}
\newtheorem{thm}[satz]{Theorem}
\newtheorem{exercise}[satz]{Aufgabe}

\begin{document}

\pagestyle{empty}
\parindent 0cm
\begin{minipage}{14cm}
  \footnotesize{\textbf{HOCHSCHULE} \hfill Dipl.-Math. Xenia Bogomolec\\
  \textbf{HANNOVER}\\
  UNIVERSITY OF\\
  APPLIED SCIENCES\\
  AND ARTS
  }
\end{minipage}
\vspace{1.0cm}

\begin{center}
  {\Large \bf Br\"uckenkurs Mathematik} \\
  \vspace{0.5cm}
  {\large L\"osungen zum \"Ubungsblatt 3 (Vektoren)}  \\
\end{center}
\vspace{0.5cm}
\normalsize
\parindent0cm

\begin{exercise}
  Berechnen Sie die L\"angen der Vektoren \\
  $\vec{u}=\left(\begin{array}{r} 3 \\ 5 \\ -12 \end{array}\right)$,
  $\vec{v}=\left(\begin{array}{r} 1 \\ 3 \\ 7 \end{array}\right)$ und
  $\vec{w}=\left(\begin{array}{r} 11 \\ 1 \\ -3 \end{array}\right)$. \\ 

  \vspace{0.1cm}

  $|\vec{u}| = \sqrt{9+25+144} = \sqrt{178}$\\
  $|\vec{v}| = \sqrt{1+9+49} = \sqrt{59}$\\
  $|\vec{w}| = \sqrt{121+1+9} = \sqrt{131}$ 

\end{exercise}

\vspace{0.1cm}

\begin{exercise}
  Gegeben seien die Vektoren
  $\vec{a}=\left(\begin{array}{r} 1 \\ 3 \end{array}\right)$,
  $\vec{b}=\left(\begin{array}{r} -2 \\ 5 \end{array}\right)$,\\
  $\vec{c}=\left(\begin{array}{r} 3 \\ 17 \end{array}\right)$ und
  $\vec{d}=\left(\begin{array}{r} 8 \\ -4 \end{array}\right)$.
  \begin{enumerate}
    \item[(a)] Berechnen und zeichnen Sie
      \begin{enumerate}
      \item[(i)] $\vec{a}+\vec{b}+\vec{c}=\left(\begin{array}{r} 2 \\ 25 \end{array}            \right)$
      \item[(ii)] $\vec{b}-\vec{d}+\vec{a}=\left(\begin{array}{r} -9 \\ 12 \end{array}              \right)$
      \item[(iii)] $\frac{1}{2}\vec{a}+\frac{1}{5}\vec{b}-3\vec{d}=\left(\begin{array}{r} -23\frac{9}{10} \\ 14\frac{1}{2} \end{array}\right)$
      \end{enumerate}
    \item[(b)] Berechnen Sie jeweils den Winkel $\varphi$ zwischen den Vektoren
      \begin{enumerate}
        \item[(i)] $\vec{a}$ und $\vec{b}:$\\
                   $\vec{a}\cdot\vec{b}=13, \,\vert\vec{a}\vert=\sqrt{10}, \, \vert\vec{b}\vert=\sqrt{29} \\ 
                   \Rightarrow \cos{\varphi}=\frac{13}{\sqrt{290}}=0,76339 \Leftrightarrow \varphi = 42,79^\circ$
        \item[(ii)] $\vec{a}$ und $\vec{d}:$\\
                    $\varphi$ ist gleich dem Winkel zwischen $\vec{a}$ und $\frac{1}{4}\vec{d}$\\
                    $\vec{a}\cdot\frac{1}{4}\vec{d}=-1, \,\vert\vec{a}\vert=\sqrt{10}, \, \vert\frac{1}{4}\vec{d}\vert=\sqrt{5} \\ 
                   \Rightarrow \cos{\varphi}=\frac{-1}{\sqrt{50}}=\frac{-1}{5\sqrt{2}} \Leftrightarrow \varphi = 98,13^\circ$
        \end{enumerate}
    \item[(c)] W\"ahlen Sie zwei Vektoren in $\mathbb{R}^2$ mit dem Zwischenwinkel $\varphi=45^\circ$. Was ist der Wert von $\cos{\varphi}$? Zeigen Sie mit der Formel aus dem Br\"uckenkurs, warum sich der Wert nicht  ver\"andert, wenn man die Vektoren mit unterschiedlichen Faktoren streckt. \\ \\
        Zum Beispiel $\vec{a}=\left(\begin{array}{r} 1 \\ 0 \end{array}\right)$ und
          $\vec{b}=\left(\begin{array}{r} 2 \\ 2 \end{array}\right)$\, \\
          $\cos{\varphi}=\frac{1}{\sqrt{2}}=\frac{\sqrt{2}}{2}$ \\ \\
          Seien $\vec{a}=\left(\begin{array}{r} a_1 \\ a_2 \end{array}\right)$,
          $\vec{b}=\left(\begin{array}{r} b_1 \\ b_2 \end{array}\right) \in \mathbb{R}^2$
          und $r,s \in \mathbb{R}$. Dann gilt:
          $\cos{\varphi}=\frac{\vec{a}\cdot\vec{b}}{\vert\vec{a}\vert \cdot \vert\vec{b}\vert}=
          \frac{a_1\cdot b_1+a_2\cdot b_2}{\sqrt{a_1^2+a_2^2}\cdot\sqrt{b_1^2+b_2^2}}
          = \frac{rs(a_1\cdot b_1+a_2\cdot b_2)}{rs\sqrt{(a_1^2+a_2^2)}\cdot\sqrt{(b_1^2+b_2^2)}}\\
          \quad=\frac{ra_1\cdot sb_1+ra_2\cdot sb_2}{\sqrt{r^2(a_1^2+a_2^2)}\cdot\sqrt{s^2(b_1^2+b_2^2)}}
          =\frac{r\vec{a}\cdot s\vec{b}}{\vert r \vec{a}\vert \cdot \vert s\vec{b}\vert}$
  \end{enumerate}
\end{exercise}

\vspace{0.1cm}
   
\begin{exercise}
  Berechnen Sie $\vec{u}\times \vec{v}$ f\"ur
  $\vec{u}=\left(\begin{array}{r} 2 \\ -1 \\ 4 \end{array}\right)$ und
  $\vec{v}=\left(\begin{array}{r} 4 \\ 3 \\ 7 \end{array}\right)$.\\
  \begin{enumerate}
    \item[]
      $\vec{u}\times \vec{v}
      =\left(\begin{array}{r} (-1)\cdot7 - 4\cdot3 \\ 4\cdot4 - 2\cdot7 \\ 2\cdot3 - (-1)\cdot 4 \end{array}\right)
      =\left(\begin{array}{r} -7-12 \\ 16-14 \\ 6+4 \end{array}\right)
      =\left(\begin{array}{r} -19 \\ 2 \\ 10 \end{array}\right) $\\
  \end{enumerate}
\end{exercise}

\vspace{0.1cm}


\begin{exercise}
  Berechnen Sie jeweils die Seitenl\"angen und den Fl\"acheninhalt des Dreiecks mit den Eckpunkten A,B und C:
  \begin{enumerate}
    \item[(a)] $A=(1,0),B=(3,5),C=(5,0):$\\ \\
              $\vec{a}=\left(\begin{array}{r} 3-1 \\ 5-0 \end{array}\right)
              =\left(\begin{array}{r} 2 \\ 5 \end{array}\right)\Rightarrow
              \,\vert\vec{a}\vert=\sqrt{29}$,\\
              $\vec{b}=\left(\begin{array}{r} 5-3 \\ 0-5 \end{array}\right)
              = \left(\begin{array}{r} 2 \\ -5 \end{array}\right)\Rightarrow
              \,\vert\vec{b}\vert=\sqrt{29}$, \\
              $\vec{c}=\left(\begin{array}{r} 1-5 \\ 0-0 \end{array}\right)
              =\left(\begin{array}{r} 4 \\ 0 \end{array}\right)\Rightarrow
              \,\vert\vec{c}\vert=\sqrt{16}=4$\\ \\
              Fl\"ache $F = \frac{1}{2}(4 \cdot 5)$ (Skizze!)
    \item[(b)] $A=(3,5),B=(-4,1),C=(5,6):$\\ \\
              $\vec{a}=\left(\begin{array}{r} -4-3 \\ 1-5 \end{array}\right)
              =\left(\begin{array}{r} -7 \\ -4 \end{array}\right)\Rightarrow
              \,\vert\vec{a}\vert=\sqrt{65}$, \\
              $\vec{b}=\left(\begin{array}{r} 5+4 \\ 6-1 \end{array}\right)
              = \left(\begin{array}{r} 9 \\ 5 \end{array}\right)\Rightarrow
              \,\vert\vec{b}\vert=\sqrt{106}$, \\
              $\vec{c}=\left(\begin{array}{r} 3-5 \\ 5-6 \end{array}\right)
              =\left(\begin{array}{r} -2 \\ -1 \end{array}\right)\Rightarrow
              \,\vert\vec{c}\vert=\sqrt{5}$\\ \\
              Erweitere zwei der Vektoren zu Vektoren in $\mathbb{R}^3$:\\ \\
              Fl\"ache $F = \frac{1}{2}\cdot\vert \left(\begin{array}{r} \vec{a}  \\ 0 \end{array}\right)\times \left(\begin{array}{r} \vec{b}  \\ 0 \end{array}\right) \vert
              =\frac{1}{2}\cdot\vert \left( \begin{array}{r} 0 \\ 0 \\ -35+36\end{array}\right) \vert 
              =\frac{1}{2}$\\ \\
              Die Fl\"ache kann mit beliebiger Vektorwahl berechnet werden, wir bekommen immer das gleiche Ergebnis!
  \end{enumerate}
\end{exercise}

\vspace{0.1cm}   

\begin{exercise}
  Berechnen Sie die Fl\"ache des durch $\vec{u}$ und $\vec{v}$ aufgespannten Dreiecks f\"ur \, \\ \\
  $\vec{u}=\left(\begin{array}{r} 1 \\ 1 \\ 1 \end{array}\right)$ und
  $\vec{v}=\left(\begin{array}{r} 4 \\ 2 \\ 4 \end{array}\right)$.\\ \\ \\
  Fl\"ache 
  $A = \frac{1}{2}\cdot\vert \left(\begin{array}{r} 1 \\ 1 \\ 1 \end{array}\right)\times \left(\begin{array}{r} 4 \\ 2 \\ 4 \end{array}\right) \vert
  = \frac{1}{2}\cdot\vert \left(\begin{array}{r} 2 \\ 0 \\ -2 \end{array}\right)\vert
  = \frac{1}{2} \sqrt{8} = \sqrt{2}$\\ \\

\end{exercise}



\end{document}
\documentclass[12pt]{article}

\usepackage{amsmath,amsthm,amsxtra,amssymb}
\usepackage{multicol}
\usepackage{rotating}
\usepackage{moreverb}
\usepackage{epsfig}

\newtheorem{satz}{Satz}
\newtheorem{thm}[satz]{Theorem}
\newtheorem{exercise}[satz]{Aufgabe}

\begin{document}

 \pagestyle{empty}
 \parindent 0cm
 \begin{minipage}{14cm}
 \footnotesize{\textbf{HOCHSCHULE} \hfill Dipl.-Math. Xenia Bogomolec\\
\textbf{HANNOVER}\\
 UNIVERSITY OF\\
 APPLIED SCIENCES\\
 AND ARTS
 }
 \end{minipage}
 \vspace{1.0cm}

\begin{center}
 {\Large \bf Br\"uckenkurs Mathematik} \\
 \vspace{0.5cm}
 {\large \"Ubungsblatt 3 (Vektoren)}  \\

\end{center}
 \vspace{0.5cm}
 \normalsize
 \parindent0cm


\begin{exercise}
  Berechnen Sie die L\"angen der Vektoren \\
  $\vec{u}=\left(\begin{array}{r} 3 \\ 5 \\ -12 \end{array}\right)$,
  $\vec{v}=\left(\begin{array}{r} 1 \\ 3 \\ 7 \end{array}\right)$ und
  $\vec{w}=\left(\begin{array}{r} 11 \\ 1 \\ -3 \end{array}\right)$.
\end{exercise}

\vspace{0.1cm}
   
 \begin{exercise}
  Gegeben seien die Vektoren
  $\vec{a}=\left(\begin{array}{r} 1 \\ 3 \end{array}\right)$,
  $\vec{b}=\left(\begin{array}{r} -2 \\ 5 \end{array}\right)$,\\
  $\vec{c}=\left(\begin{array}{r} 3 \\ 17 \end{array}\right)$ und
  $\vec{d}=\left(\begin{array}{r} 8 \\ -4 \end{array}\right)$.
  \begin{enumerate}
  \item[(a)] Berechnen und zeichnen Sie
      \begin{enumerate}
      \item[(i)] $\vec{a}+\vec{b}+\vec{c}$
      \item[(ii)] $\vec{b}-\vec{d}+\vec{a}$
      \item[(iii)] $\frac{1}{2}\vec{a}+\frac{1}{5}\vec{b}-3\vec{d}$
      \end{enumerate}
  \item[(b)] Berechnen Sie jeweils den Winkel $\varphi$ zwischen den Vektoren
    \begin{enumerate}
      \item[(i)] $\vec{a}$ und $\vec{b}$
      \item[(ii)] $\vec{a}$ und $\vec{d}$
      \end{enumerate}
    \item[(c)] W\"ahlen Sie zwei Vektoren in $\mathbb{R}^2$ mit dem Zwischenwinkel $\varphi=45^\circ$. Was ist der Wert von $\cos{\varphi}$? Zeigen sie mit der Formel aus dem Br\"uckenkurs, warum sich der Wert nicht  ver\"andert, wenn man die Vektoren mit unterschiedlichen Faktoren streckt. 
  \end{enumerate}
   \end{exercise}

\vspace{0.1cm}   


\begin{exercise}
  Berechnen Sie $\vec{u}\times \vec{v}$ f\"ur
  $\vec{u}=\left(\begin{array}{r} 2 \\ -1 \\ 4 \end{array}\right)$ und
  $\vec{v}=\left(\begin{array}{r} 4 \\ 3 \\ 7 \end{array}\right)$ .
\end{exercise}

\vspace{0.1cm} 
   

\begin{exercise}
  Berechnen Sie jeweils die Seitenl\"angen und den Fl\"acheninhalt des Dreiecks mit den Eckpunkten A,B und C:
    \begin{enumerate}
    \item[(a)] $A=(1,0),B=(3,5),C=(5,0)$
    \item[(b)] $A=(3,5),B=(-4,1),C=(5,6)$
    \end{enumerate}
\end{exercise}

\vspace{0.1cm} 

\begin{exercise}
  Berechnen Sie die Fl\"ache des durch $\vec{u}$ und $\vec{v}$ aufgespannten Dreiecks f\"ur \,
  $\vec{u}=\left(\begin{array}{r} 1 \\ 1 \\ 1 \end{array}\right)$ und
  $\vec{v}=\left(\begin{array}{r} 4 \\ 2 \\ 4 \end{array}\right)$.
\end{exercise}
   
 
   
\end{document}